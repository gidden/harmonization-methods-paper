\section{Introduction}

Integrated Assessment Models (IAMs) are tools used to understand the complex
interactions between energy systems, economic systems, land use, the climate,
and air pollution, among other sectors . IAMs provide global projections of
systemic change by dividing the world into a number of representative regions
(typically 10 to 30), the definition of which is distinct for each model
\cite{krey_global_2014}. Results from IAMs are integral in a number of
international studies, which notably include projections of climate and economic
futures. Recently, flagship IAMs have developed scenarios based on the Shared
Socioeconomic Pathways (SSPs) \cite{van_vuuren_energy_2017, fricko_marker_2017,
  fujimori_ssp3:_2017, calvin_ssp4:_2017, kriegler_fossil-fueled_2017} which
quantify a variety of potential global futures. The SSPs are primarily designed
to be used in analyses of broad climate research which include earth system
model (ESM) simulations, climate impact, adaptation climate mitigation
studies. They consequently underpin the United Nations' Intergovernmental Panel
on Climate Change (IPCC) across working groups.

While IAMs are implemented in myriad ways, including simulation and
optimization, the core inputs, outputs, and purpose are similar across different
models. Modeling teams incorporate existing data sets of energy systems, land
use, economics, demographics and emissions sources and concentrations, among other
data, in order to provide consistent historical trajectories of modeled
variables. The models then provide estimates of future trajectories of these
variables under various socioeconomic and technological assumptions as well as
proposed policy constraints, e.g., targets for future Greenhouse Gas (GHG)
emissions.

The emissions trajectories calculated by IAMs are critical inputs for ongoing,
worldwide scientific community efforts in the Coupled Model Intercomparison
Project Phase 6 (CMIP6) \cite{eyring_overview_2016}, which is utilizing a number
of marker SSP scenarios developed by the IAM community
\cite{oneill_scenario_2016}. These trajectories are endogenously calculated by
modeling the individual technologies and sectors that contribute towards the
emissions of different air pollutants and GHGs as well as various mitigation
technologies (e.g., NOx and SOx scrubbers on power plants). However, the
historical emissions inventories used by different models can differ by large
amounts depending on the region, sector, and emissions species being modeled.

In practice, IAMs calculate the total source intensity of emitting technologies,
for example the total activity of coal power plants in China, and incorporate
emissions-intensity factors for individual gas species, for example the quantity
of sulfur emissions from coal plants per megawatt-hour of production, in order
to determine total emissions of a given gas species from a given
technology. Models are then \textit{calibrated} to historical data sources which
provide both the activity of emitting technologies as well as trajectories of
past emissions in order to determine historical emissions factors in each model
region for each technology.

With some frequency, however, historical data is updated to incorporate new
assumptions of source activity, new emissions accounting practices, or other
concerns. The global climate change community has recently developed a standard
historical emissions dataset for both anthropogenic emissions (i.e., the
Community Emissions Datasystem (CEDS)) \cite{hoesly_historical_2017} and
land-use change (LUC) emissions \cite{van_marle_historic_2017} which, in
conjunction with the SSP IAM trajectories, will be used for climate-related
modeling exercises of CMIP6.

Because the process of model calibration is quite onerous, model teams initially
incorporate updated historical datasets through the process of
\textit{harmonization} in order to guarantee consistency with these new
historical datasets. Harmonization describes the process of updating model
results to incorporate new historical time series while maintaining critical
aspects of modeled future trajectories impacted outputs. In the emissions
context, this means that each individual combination of model region, model
sector, and emissions species must be harmonized. Depending on the total number
of model regions, sectors, and emissions species, this can require the selection
of thousands to tens-of-thousands of harmonization methods.

Harmonization has been addressed in previous studies as it is a common practice
in the IAM and climate change communities. For example,
\cite{meinshausen_rcp_2011} describes the use of scaling routines for the 5
regions used in the Special Report on Emissions Scenarios (SRES)
\cite{nakicenovic2000}; however, only total emissions were harmonized in the
exercise, thus there is no sectoral dimension. Further,
\cite{rogelj_discrepancies_2011} describes the impacts of choosing various
harmonization routines on future trajectories. Importantly, the choice of
harmonization method to date has been determined via expert solicitation and has
generally been applied to all trajectories for a given class of emissions
species.

In recent studies, i.e., the Representative Concentration Pathways (RCPs), IAM
results have been harmonized by sector and the 5 RCP global regions
\cite{vuuren_representative_2011}. Climate modeling efforts have continued to
progress, however, demanding increased spatial and sectoral resolution from
IAMs. In order to address the growing dimensionality of model outputs while
still providing a consistent and scientifically rigorous harmonization
procedure, an automated process for determining harmonization methods is
needed. The remainder of this paper describes the methodology of the
\code{aneris} code base \cite{matthew_gidden_2017_802832}, written in the Python
programming language (detailed documentation available
\href{http://mattgidden.com/aneris/}{online}), which implements such a system in
Section \ref{sec:meths}. The results of applying the automated harmonization
mechanism on two example IAM scenarios, one with emissions growth and another
with emissions mitigation, is presented in Section \ref{sec:results}. Finally,
the general effectiveness and potential future improvements on the automated
methodology is discussed in Section \ref{sec:future}.

% todo: add?
% what aspects of harmonization are important (e.g., cumulative carbon emissions).
