\section{Introduction}

Integrated Assessment Models (IAMs) are tools used to understand the complex
interactions between energy systems, economic systems, land use, the climate,
and air pollution, among other sectors . IAMs provide global projections of
% * <sfujimori112256@gmail.com> 2017-05-24T20:16:51.179Z:
% 
% Here, Volker's paper seems well fit for the citation.
% http://onlinelibrary.wiley.com/doi/10.1002/wene.98/abstract
% 
% 
% ^.
systemic change by dividing the world into a number of representative regions (typically 10 to 30),
the definition of which is distinct for each model. Results from IAMs are
integral in a number of international studies, which notably include projections
of climate and economic futures. Recently, flagship IAMs have developed
scenarios based on the Shared Socioeconomic Pathways (SSPs)
\cite{van_vuuren_energy_2017, fricko_marker_2017, fujimori_ssp3:_2017,
  calvin_ssp4:_2017, kriegler_fossil-fueled_2017} which quantify a variety of
potential global futures. The SSPs are primarily designed to being used in analyses of broad climate research which includes earth system model (ESM) simulations, climate impact, adaptation climate mitigation studies. They consequently underpin the
Intergovernmental Panel on Climate Change (IPCC) across working groups and the United Nations
% * <sfujimori112256@gmail.com> 2017-05-24T20:32:22.989Z:
% 
% IPCC itself collects, reviews and summarizes papers. SSPs are officially independent activity from IPCC. So, I slightly changed the context. 
% 
% ^.
% * <sfujimori112256@gmail.com> 2017-05-24T20:37:11.995Z:
% 
% I think Paris agreement is not so relevant with SSPs so I removed.
% 
% ^.

While IAMs are implemented in myriad ways, including simulation and
optimization, the core inputs, outputs, and purpose are similar across different
models. Modeling teams incorporate existing datasets of energy systems, land use, economic, demographic and emissions sources and concentrations, among other
data, in order to provide consistent historical trajectories of modeled
variables. The models then provide estimates of future trajectories of these
variables under various socioeconomic and technological assumptions as well as
proposed policy constraints, e.g., targets for future Greenhouse Gas (GHG)
emissions.

Future emissions pathways are of critical importance to both of global and local
% * <sfujimori112256@gmail.com> 2017-05-24T20:47:32.377Z:
% 
% > Future emissions pathways are of critical importance to both of global and local
% > governance concerns. For example, the United Nation's Sustainable Development
% > Goals (SDGs) \TODO{CITE} speak directly to the impacts of emissions in Goal 3
% > (Good Health and Well-Being) and Goal 13 (Climate Action). IAMs endogenously
% > calculate the trajectories many of the emissions species of import that impact
% > the progress towards these SDGs by modeling the individual technologies and
% > sectors that contribute towards the emissions of different air pollutants and
% > GHGs as well as various mitigation technologies (e.g., NOx and SOx scrubbers on
% > power plants).
% I am not sure whether this paragraph is needed, because if we only discuss about SDGs, the emissions harmonization is not the necessity process for us essentially.
% 
% I would directly keen much more on the needs of harmonization and its importance. Then, the logic would be 
% 1) SSPs will be an basis for CMIP6 (Brian's scenarioMIP GMD paper citation)
% 2) CMIP6 uses different model outcomes (e.g. SSP1-26 IMAGE, SSP2-45 MESSAGE), which means we need somehow harmonized data set for at least base year.
% 3) However, current IAMs emissions inventory differ a lot, which is hardly acceptable for CMIP experiments.
% 4) 
% 
% 
% 
% ^.
governance concerns. For example, the United Nation's Sustainable Development
Goals (SDGs) \TODO{CITE} speak directly to the impacts of emissions in Goal 3
(Good Health and Well-Being) and Goal 13 (Climate Action). IAMs endogenously
calculate the trajectories many of the emissions species of import that impact
the progress towards these SDGs by modeling the individual technologies and
sectors that contribute towards the emissions of different air pollutants and
GHGs as well as various mitigation technologies (e.g., NOx and SOx scrubbers on
power plants).

In practice, IAMs calculate the total source intensity of emitting technologies,
for example the total activity of coal power plants in China, and incorporate
emissions-intensity factors for individual gas species, for example the quantity
of sulfur emissions from coal plants per megawatt-hour of production, in order
to determine total emissions of a given gas species from a given
technology. Models are then \textit{calibrated} to historical data sources which
provide both the activity of emitting technologies as well as trajectories of
past emissions in order to determine historical emissions factors in each model
region for each technology.

With some frequency, however, historical data is updated to incorporate new
assumptions of source activity, new emissions accounting practices, or other
concerns. The global climate change community has recently developed a standard
historical emissions dataset for both anthropogenic emissions (i.e., the
Community Emissions Datasystem (CEDS)) \cite{hoesly_historical_2017} and
land-use change (LUC) emissions \cite{van_marle_historic_2017} as part of the
6th phase of the Coupled Model Inter-comparison Project (CMIP6). These
historical emissions trajectories in conjunction with future trajectories as
provided by IAMs are critical inputs to the climate-related modeling exercises
of CMIP6.

Because the process of model calibration is quite onerous, model teams initially
incorporate updated historical datasets through the process of
\textit{harmonization} in order to guarantee consistency with these new
historical datasets. Harmonization describes the process of updating model
results to incorporate new historical time series while maintaining critical
aspects of modeled future trajectories impacted outputs. In the emissions
context, this means that each individual combination of model region, model
sector, and emissions species must be harmonized. Depending on the total number
of model regions, sectors, and emissions species, this can require the selection
of thousands to tens-of-thousands of harmonization methods.

Harmonization has been addressed in previous studies as it is a common practice
in the IAM and climate change communities. For example,
\cite{meinshausen_rcp_2011} describes the use of scaling routines for the 5
regions used in the Special Report on Emissions Scenarios (SRES)
\cite{nakicenovic2000}; however, only total emissions were harmonized in the
exercise, thus there is no sectoral dimension. Further,
\cite{rogelj_discrepancies_2011} describes the impacts of choosing various
harmonization routines on future trajectories. Importantly, the choice of
harmonization method to date has been determined via expert solicitation and has
generally been applied to all trajectories for a given class of emissions
species.

In order to incorporate the large variety of possible model results and given
% * <sfujimori112256@gmail.com> 2017-05-24T21:09:01.445Z:
% 
% I think it would be better to declare what metrics we are going to use to evaluate the methodology. In what case we can say this method succeeds or fails? That would be essentially the one of the highlight of the conclusion.
% 
% ^.
relatively large increases in regional and sectoral dimensionality, an automated
process for determining harmonization methods is needed. The remainder of this
% * <sfujimori112256@gmail.com> 2017-05-24T21:12:03.999Z:
% 
% > regional and sectoral dimensionality, an automated
% > process for determining harmonization methods is needed. The remainder of this
% 
% I suggest to add some explanations why detailed regional and sectoral harmonization is needed. As you mentioned, from RCPs sectoral and regional harmonization is needed. Meantime, ESM (or in this case we should say climate models) have been increasing the spatial resolution overtime and much more detailed representation become available, which essentially requires detailed information. The obvious example is AerChemMIP.
% 
% ^.
paper describes the methodology of the \code{aneris} code base \TODO{cite},
written in the Python programming language, which implements such a system in
Section \ref{sec:meths}. The results of applying the automated harmonization
mechanism on two example IAM scenarios, one with emissions growth and another
with emissions mitigation, is presented in Section \ref{sec:results}. Finally,
the general effectiveness and potential future improvements on the automated
methodology is discussed in Section \ref{sec:future}.

% todo: add?
% what aspects of harmonization are important (e.g., cumulative carbon emissions).