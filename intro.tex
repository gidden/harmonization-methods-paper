\section{Introduction}

Integrated Assessment Models (IAMs) are tools used to understand the complex
interactions between energy systems, economoic systems, land use, the climate,
and air pollution, among other sectors \cite{van_vuuren_energy_2017,
  kriegler_fossil-fueled_2017, calvin_ssp4:_2017, fujimori_ssp3:_2017,
  fricko_marker_2017}. IAMs provide global projections of systemic change by
dividing the world into a number of representative regions, the definition of
which is distinct for each model. Results from IAMs are integral in a number of
international studies, which notably include projections of climate and economic
futures used in analyses underpinning the Intergovenmental Panel on Climate
Change (IPCC) \TODO{cite} and the United Nations Framework Convention on Climate
Unchange (UNFCCC)'s Paris Agreement \TODO{cite}.

While IAMs are implemented in myriad ways, including simulation and
optimization, the core inputs, outputs, and purpose are similar across different
models. Modeling teams incorporate existing datasets of energy systems, world
development indicators, and emissions sources and concentrations, among other
data, in order to provide consistent historical trajectories of modeled
variables. The models then provide estimates of future trajectories of these
variables under various socioeconomic and technological assumptions as well as
proposed policy constraints, e.g., targets for future Greenhouse Gas (GHG)
emissions.

Future emissions pathways are of critical importance to both of global and local
governance concerns. For example, the United Nation's Sustainable Development
Goals (SDGs) \TODO{CITE} speak directly to the impacts of emissions in Goal 3
(Good Health and Well-Being) and Goal 13 (Climate Action). IAMs endogenously
calculate many of the emissions species of import that impact the progress
towards these SDGs by modeling the individual technologies and sectors that
contribute towards the emissions of different air pollutants and GHGs as well as
various mitigation technologies (e.g., NOx and SOx scrubbers on power plants)
that are used due to regulation or implemented policies.

In practice, IAMs calculate the total source intensity of emitting technologies,
for example the total activity of coal power plant emissions in China, and
incorporate emissions-intensity factors for individual species, for example the
quantity of sulfur emissions from coal plants per megawatt-hour of production,
in order to determine total emissions of a given gas species from a given
technology. Models are then \textit{calibrated} to historical data sources which
provide both the activity of emitting technologies as well as trajectories of
past emissions in order to determine historical emissions factors in each model
region for each technology.

With some frequency, however, historical data is updated to incorporate new
assumptions of source activity, new emissions accounting practices, or other
concerns. The global climate change community has recently updated a standard
historical emissions dataset for both anthropogenic and land-use change
emissions \TODO{cite ceds and LUC paper} as part of the 6th phase of the Coupled
Model Intercomparison Project \TODO{cite cmip6} (CMIP6). These historical
emissions trajectories in conjunction with future trajectories as provided by
IAMs are critical inputs to the climate-related modeling exercises of CMIP6.

Because the process of model calibration is quite onerous, model teams initially
incorporate updated historical datasets through the process of
\textit{harmonization} in order to guarantee consistency with new historical
datasets. Harmonization describes the process of updating model results to
incorporate new historical timeseries while maintaining critical aspects of
modeled future trajectories for variables of import. In the emissions context,
this means that each individual combination of model region, model sector, and
emissions species must be harmonized. Depending on the model, this can require
the selection of thousands to tens-of-thousands of harmonization methods.

Harmonization has been addressed in previous studies as it is a common practice
in the IAM and climate change communities. For example,
\cite{meinshausen_rcp_2011} describes a harmonization process using scaling
routines for the 5 SRES regions with no sectoral definition. Further,
\cite{rogelj_discrepancies_2011} describes the impacts of choosing various
harmonization routines on future trajectories. Importantly, the choice of
harmonization method to date has been determined via expert solicitation and has
generally been applied to all trajectories for a given class of emissions
species.

In order to incoporate the variety of possible model results and given
relatively large increases in regional and sectoral dimensionality, an automated
process for determining harmonization methods is needed. The remainder of this
paper describes the methodology of the \code{aneris} codebase which implements
such a system in Section \ref{sec:meths}. The results of applying the automated
harmonization mechanism on two example IAM scenarios, one with emissions growth
and another with emissions mitigation, is presented in Section
\ref{sec:results}. Finally, the general effectiveness and potential future
improvements on the automated methodology is discussed in Section
\ref{sec:future}.

% todo: add?
% what aspects of harmonization are important (e.g., cumulative carbon emissions).
