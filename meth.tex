\section{Methodology \& Implementation}\label{sec:meths}

\subsection{Harmonization Methods}

IAM emission results are provided along a number of different dimensions,
including, temporal (normally half decade or decade), spatial (i.e., model
regions), gas species, and sectoral. Each individual temporal trajectory (i.e.,
unique combinations of regions, species, and sectors) must be harmonized to the
initial time period. Given a model trajectory, $m_{r, g, s}(t)$, historical
trajectory, $h_{r, g, s}(t)$, and initial time period $t_i$, a harmonized
trajectory can be calculated.

In previous studies, two \textit{families} of methods have been used: those that
operate on the ratio of base year values (i.e., $\frac{h(t_i)}{m(t_i)}$) and
those that operate on the offset of base year values (i.e., $h(t_i) -
m(t_i)$). The \code{aneris} code base provides a number of variations on the
classic methods, including ratio-convergence methods shown in Equation
\ref{eqs:ratio}, offset-convergence methods shown in Equation \ref{eqs:offset},
and interpolation methods shown in Equation \ref{eqs:interpolate}; the
convergence factor for associated methods is shown first in Equation
\ref{eqs:factor}. Each is a function of time, model trajectory, historical
trajectory, harmonization time period, and a convergence time period ($t_f$),
where the harmonized trajectory converges to the unharmonized trajectory.

\begin{equation}\label{eqs:factor}
  \beta(t, t_i, t_f) =
  \begin{cases}
    1 - \frac{t - t_i}{t_f - t_i},& \text{if } t \leq t_f\\
    0,                        & \text{otherwise}
  \end{cases}
\end{equation}

\begin{equation}\label{eqs:ratio}
  m^{rat}(t, m, h, t_i, t_f) = [\beta(t, t_i, t_f) (\frac{h(t_i)}{m(t_i)} - 1) + 1] m(t)
\end{equation}

\begin{equation}\label{eqs:offset}
  m^{off}(t, m, h, t_i, t_f) = \beta(t, t_i, t_f) (h(t_i) - m(t_i)) + m(t)
\end{equation}
  
\begin{equation}\label{eqs:interpolate}
  m^{int}(t, m, h, t_i, t_f) =
  \begin{cases}
    \frac{m(t_f) - h(t_i)}{t_f - t_i}(t - t_i) + h(t_i), & \text{if } t \leq t_f\\
    m(t), & \text{otherwise}
  \end{cases}
\end{equation}

The \code{aneris} code base provides a family of methods to choose from for each
of the harmonization cases. A summary of all available methods is provided below
in Table \ref{tab:meths}.

\begin{table}[]
\centering
\caption{All Harmonization Methods Provided in \code{aneris}}
\label{tab:meths}
\begin{tabular}{|l|l|l|}
\hline
Method Name                             & Harmonization Type & Convergence Year\\
\hline
\code{constant\_ratio}                  & ratio              & $t_f = \infty$\\
\code{reduce\_ratio\_<year>}            & ratio              & $t_f = $\code{<year>}\\
\code{constant\_offset}                 & offset             & $t_f = \infty$\\
\code{reduce\_offset\_<year>}           & offset             & $t_f = $\code{<year>}\\
\code{linear\_interpolate\_<year>}      & interpolation      & $t_f = $\code{<year>}\\
\hline
\end{tabular}
\end{table}

\subsection{Default Method Decision Tree}\label{sec:tree}

While historically expert opinion has guided the determination of harmonization
methods to choose for a given trajectory, a new, \textit{decision tree} approach
has been implemented in this work. In order to provide reasonable
\textit{default} methods, the historical trajectory, unharmonized model
trajectory, and relative difference between history and model values in the
harmonization year are analyzed. A figure of the decision tree used in this
analysis is shown in Figure \ref{fig:decision_tree}. 


\begin{figure}
  \begin{center}
    \includegraphics[width=\textwidth]{decision_tree.png}
    \caption[]{
      \label{fig:decision_tree}
      The default method decision tree used in the \code{aneris} software
      library. Decision-making threshold are described below.}
  \end{center}
\end{figure}

A number of characteristics impact the decision of which default method to
select based on the effect of the characteristic on the potential harmonized
trajectory. For example, it is possible for models to report zero values in the
harmonization year in situations in which technologies are introduced in future
time periods in regions or for sectors which produce an emissions species that
is absent in the initial modeling period. In such cases, an offset method is
required as a ratio method would mask future emissions and erroneously harmonize
the trajectory. In most cases, however, models do report values in the
harmonization year. When model and historical values are relatively close, a
convergence method is provided in order to be as representative as possible to
the underlying unharmonized model results. The final model-based characteristic
of import is whether the model reports negative emissions. Such a case is
possible for gas species which can be extracted from the environment, as is the
case for CO2 in future scenarios with high mitigation strategies. If a model
provides a trajectory that transitions from positive to negative emissions, then
a convergence method is used in order to guarantee capture of this
transition. Otherwise, it is possible, given the relative values between history
and model in the harmonization year, for a negative trajectory to be
inappropriately harmonized to positive, but decreasing, trajectory.

Variability of the historical trajectory is also an important characteristic
when considering the choice of harmonization method. Such trajectories are
normally indicative of emissions from land use sectors. Land use sectors are
modeled with varying fidelity by IAMs, but in general proxies are used for
emissions trajectories. Take for example the emissions of black carbon (BC) from
grassland burning in Africa \TODO{check that this is a good sector}: for most
models, the use of grass and paturelands is modeled, and the emissions of
species from the burning of these lands is determined by applying a emissions
factor to a proxy model of burning of these lands; in general, the more
grassland usage occurs, the higher the emissions from burning of
grasslands. Therefore, consistency in harmonization method is important due to
the approaches taken in modeling these sectors by IAMs. In order to reliably
detect land-use-like emissions, a measure of the co-variance of the historical
trajectory is calculated as shown in Equation \ref{eqs:cov} and tested against a
threshold. To determine this threshold, an analysis of the recent CEDS
\TODO{cite ceds paper} historical data has been performed. Figure \ref{fig:cov}
shows the distribution of land-use co-variances and non-land-use co-variances as
determined for historical data aggregated to the model regions of 5 different
flagship IAMs \TODO{cite all 5 IAMs}. A threshold value of 20 has been chosen
based on these observations. Importantly, tails of the LUC and non-LUC overlap,
thus there are both false positives (~10\% of non-LUC trajectories) and false
negatives (~7\% of LUC trajectories). However, as any regional definition is
model dependent and thus any regional aggregation is possible an automated
detection mechanism is necessary.

\begin{equation}\label{eqs:cov}
    \Omega =  \frac{\sigma(h^{\prime}(t))}{\mu(h^{\prime}(t))}
\end{equation}


\begin{figure}
  \begin{center}
    \includegraphics[width=\textwidth]{cov.pdf}
    \caption[]{
      \label{fig:cov}
      The distribution of $\Omega$ values for LUC and non-LUC historical
      trajectories is shown. CEDS historical data is used for non-LUC data and
      \TODO{cite LUC data} is used for LUC data. All historical data has been
      aggregated from countries to IAM model regional definitions, and all gas
      species included in the historical datasets are included in the
      analysis. The solid black line indicates the threshold value used by
      default in \code{aneris}.  
    }
  \end{center}
\end{figure}

Finally, consideration is taken with respect to the relative difference between
the historic and model values in the harmonization year. In order to investigate
the possible values that these relative differences can take, the flagship model
values used in the SSP \TODO{CITE} and (ongoing) CMIP6 inter-comparison exercises
are used. A distribution of these differences for all models in the study is
presented in Figure \ref{fig:dh}. Given the available data, a threshold value of
50\% was chosen to be used as a default in \code{aneris}.

\begin{figure}
  \begin{center}
    \includegraphics[width=\textwidth]{dh.pdf}
    \caption[]{
      \label{fig:dh}
      The distribution of relative differences between model and historical
      values in the harmonization year is shown. The solid black line indicates
      the threshold value used by default in \code{aneris}.  
    }
  \end{center}
\end{figure}

\subsection{\code{aneris} Workflow}

The full \code{aneris} workflow is comprised of a number of components shown
graphically in Figure \ref{fig:workflow}. Unharmonized model data and
run-control configuration are read in via an Excel spreadsheet. Data is assumed
to be in the \textit{IAM Consortium} format, i.e., using \code{Model},
\code{Scenario}, \code{Region}, \code{Variable}, and \code{Unit} columns in
addition to column representing each modeled time period. Users are able to
control the harmonization process via a number of controls. First, users can
provide \textit{override} methods for any combination of region and variable. If
override methods are provided, they are used instead of the default methods as
determined by the default method decision tree. Further, users can set the
above-mentioned thresholds as well as the LUC method used in the decision
tree. Further detail of input parameters can be found at
\hyperlink{http://aneris.readthedocs.io/en/latest/}.

\begin{figure}
  \begin{center}
    \includegraphics[height=0.85\textheight]{aneris-workflow.pdf}
    \caption[]{
      \label{fig:workflow}
      The full harmonization process as executed by \code{aneris}. Operations
      that can be configured with user-based input configurations are shown in
      purple. The core harmonization process is shown in orange.  }
  \end{center}
\end{figure}

Input data then undergoes a cleaning operation, which adds any trajectories in
the historical dataset but not provided by the model and detects any issues that
would cause the harmonization process to fail. The methods used to harmonize the
data are then determined and the harmonization process is executed. Upon
completion of the harmonization process, aggregation of common analysis regions
is performed (e.g., the 5 RCP regions \TODO{CITE}). Finally, any exogenous
trajectories the user provides are added. Exogenous trajectories are normally
provided for unmodeled gases with well-accepted scenario trajectories, e.g.,
Chlorofloro Carbons provided by WMO \TODO{CITE}. Upon completion, the harmonized
trajectories and metadata regarding the harmonization process are returned. A
description of all returned metadata is provided in Table \ref{tab:metadata}.

\begin{table}[]
\centering
\caption{Metadata provided by the \code{aneris} harmonization routine. This metadata is provided for every combination of region, sector, and emissions species.}
\label{tab:metadata}
\begin{tabular}{|p{2cm}|p{8cm}|}
\hline
Column       & Description                                                                  \\
\hline
\hline
method       & The harmonization method used.                                               \\
\hline
default      & The default harmonization method as determined by the default decision tree. \\
\hline
override     & The method provided as an override (if any).                                 \\
\hline
offset       & The offset value between history and model in the harmonization year.        \\
\hline
ratio        & The ratio value between history and model in the harmonization year.         \\
\hline
cov          & The covariance value of the historical trajectory.                           \\
\hline
unharmonized & The unharmonized value in the harmonization year.                            \\
\hline
history      & This historical value in the harmonization year.                             \\
\hline
harmonized   & The resulting harmonized value in the harmonization year.\\
\hline
\end{tabular}
\end{table}


