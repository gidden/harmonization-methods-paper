\begin{frontmatter}

\title{A Methodology and Implementation of Automated Emissions Harmonization for Use in Integrated Assessment Models}

%% Group authors per affiliation:
\author[iiasa]{Matthew J. Gidden\corref{corref}}
\ead{gidden@iiasa.ac.at}
\cortext[corref]{Corresponding author}

\author[iiasa,nies]{Shinichiro Fujimori}

\address[iiasa]{International Institute for Applied Systems Analysis,
  Schlossplatz 1, A-2361 Laxenburg, Austria}
\address[nies]{National Institute for Environmental Studies, Tsukuba, Japan}

\begin{abstract}
Harmonization describes the process of calibrating Integrated Assessment Model
(IAM) GHG and air pollutants results with a new source of historical emissions trajectories. To date,
harmonization has been performed separately by individual modeling teams through
% * <sfujimori112256@gmail.com> 2017-05-24T19:26:26.142Z:
% 
% > harmonization has been performed separately by individual modeling teams 
% I think harmonization is not done. We just report the direct model outputs.
% 
% One of the ways to  make the background is that earlier IPCC scenarios (SRES and RCPs) did not document the harmonization method and far from transparent.
% 
% ^.
the use of expert opinion and solicitation. As models and their results become
more complex in both regional and sectoral dimensions, an automated approach to
choosing harmonization methods becomes necessary. This work describes a novel
% * <sfujimori112256@gmail.com> 2017-05-24T20:13:08.493Z:
% 
% > novel
% It would be better to stress how novel and useful.
% 
% ^.
methodology for determining such harmonization methods and an open-source Python
code base for implementing the methodology. Results are shown for two example
scenarios using the MESSAGE-GLOBIOM IAM that satisfactorily harmonize over 98\%
of the total emissions trajectories.
\end{abstract}

\begin{keyword}
Integrated Assessment Modeling, Harmonization, Climate Change, Air Pollution 
\end{keyword}

\end{frontmatter}

\linenumbers