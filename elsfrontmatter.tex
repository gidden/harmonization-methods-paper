\begin{frontmatter}

\title{A Methodology and Implementation of Automated Emissions Harmonization for Use in Integrated Assessment Models}

%% Group authors per affiliation:
\author[iiasa]{Matthew J. Gidden\corref{corref}}
\ead{gidden@iiasa.ac.at}
\cortext[corref]{Corresponding author}

\author[nies]{Shinichiro Fujimori}
\author[pbl]{Maarten van den Berg}
\author[pik]{David Klein}
\author[pnnl]{Steven J. Smith}
\author[pbl]{Detlef P. van Vuuren}
\author[iiasa]{Keywan Riahi}

\address[iiasa]{International Institute for Applied Systems Analysis,
  Schlossplatz 1, A-2361 Laxenburg, Austria}
\address[nies]{Center for Social and Environmental Systems Research, National Institute for Environmental Studies, 16-2 Onogawa, Tsukuba, Ibaraki 305-8506, Japan}
\address[pbl]{PBL Netherlands Environmental Assessment Agency, Postbus 30314, 2500 GH The Hague, Netherlands}
\address[pik]{Potsdam Institute for Climate Impact Research (PIK), Member of the Leibniz Association, P.O. Box 60 12 03, D-14412 Potsdam, Germany}
\address[pnnl]{Joint Global Change Research Institute, 5825 University Research Court, Suite 3500, College Park, MD 20740}

\begin{abstract}
Emission harmonization refers to the process used to match results from
Integrated Assessment Model (IAM) GHG and air pollutants against a common source
of historical emissions trajectories. To date, harmonization has been performed
separately by individual modeling teams. For the hand-over of emission data for
the Shared Socio-economic Pathways (SSPs) to climate model groups, a new
automated automatized approach was developed and used based upon commonly agreed
upon algorithms. This work describes a novel methodology for determining such
harmonization methods and an open-source Python software library implementing
the methodology. Results are shown for two example scenarios (with and without
climate policy cases) using the MESSAGE-GLOBIOM IAM that satisfactorily
harmonize over 96\% of the total emissions trajectories while having a
negligible effect on key long-term climate indicators. This new capability
enhances the comparability across different models, increases transparency and
robustness of results, and allows other teams to easily participate in
intercomparison exercises by using the same harmonization mechansim.
\end{abstract}

\begin{keyword}
Integrated Assessment Models, Climate Change, Harmonization, Air Pollution 
\end{keyword}

\end{frontmatter}

\linenumbers
