\begin{frontmatter}

\title{A Methodology and Implementation of Automated Emissions Harmonization for Use in Integrated Assessment Models}

%% Group authors per affiliation:
\author[iiasa]{Matthew J. Gidden\corref{corref}}
\ead{gidden@iiasa.ac.at}
\cortext[corref]{Corresponding author}

\author[nies]{Shinichiro Fujimori}
\author[pbl]{Maarten van den Berg}
\author[pik]{David Klein}
\author[pnnl]{Steven J. Smith}
\author[iiasa]{Keywan Riahi}

\address[iiasa]{International Institute for Applied Systems Analysis,
  Schlossplatz 1, A-2361 Laxenburg, Austria}
\address[nies]{National Institute for Environmental Studies, Tsukuba, Japan}
\address[pbl]{PBL Netherlands Environmental Assessment Agency, Postbus 30314, 2500 GH The Hague, Netherlands}
\address[pik]{Potsdam Institute for Climate Impact Research (PIK), Member of the Leibniz Association, P.O. Box 60 12 03, D-14412 Potsdam, Germany}
\address[pnnl]{Joint Global Change Research Institute, 5825 University Research Court, Suite 3500, College Park, MD 20740}

\begin{abstract}
Harmonization describes the process required to match Integrated Assessment
Model (IAM) GHG and air pollutants results against a common source of historical
emissions trajectories. To date, harmonization has been performed separately by
individual modeling teams through
% * <maarten.vandenberg@pbl.nl> 2017-07-03T20:19:43.403Z:
% 
% IMAGE has provided harmonized emissions in the past, generated by a harmonization routine applying a ratio convergence method (on a higher aggregation level) (Hof et al., 2016). I think the contrast with these harmonization approaches and this work, which enables a documented, automated process selecting appropriate harmonization methodologies, might not be clear. Furthermore, later in the text cross-model harmonization methods (though less sophisticated) are mentioned. It is worth mentioning that previous approaches are far less sophisticated.
% 
% ^.
% * <sfujimori112256@gmail.com> 2017-05-24T19:26:26.142Z:
% 
% > harmonization has been performed separately by individual modeling teams 
% I think harmonization is not done. We just report the direct model outputs.
% 
% One of the ways to  make the background is that earlier IPCC scenarios (SRES and RCPs) did not document the harmonization method and far from transparent.
% 
% ^.
the use of expert opinion and solicitation. In order to support ongoing and
future global climate and socioeconomic analyses, incorporate model advancements
towards more complex regional and sectoral representations, and expand the
capability of modeling teams to participate in intercomparison exercises, an
automated approach to choosing harmonization methods becomes necessary. This
work describes a novel
% * <sfujimori112256@gmail.com> 2017-05-24T20:13:08.493Z:
% 
% > novel
% It would be better to stress how novel and useful.
% 
% ^.
methodology for determining such harmonization methods and an open-source Python
code base for implementing the methodology. Results are shown for two example
scenarios (with and without climate policy cases) using the MESSAGE-GLOBIOM IAM that satisfactorily harmonize over 96\%
of the total emissions trajectories.
\end{abstract}

\begin{keyword}
Integrated Assessment Modeling, Harmonization, Climate Change, Air Pollution 
\end{keyword}

\end{frontmatter}

\linenumbers
