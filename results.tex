\section{Results \& Conclusions}

\subsection{A Harmonization Test Case}

In order to test the \code{aneris} harmonization procedure, results from the IAM
MESSAGE-GLOBIOM \TODO{CITE} was processed. Two scenarios from the Shared Socioeconomic
Pathways scenario library \TODO{CITE} are investigated. The reference SSP2, or
``middle of the road'' is chosen first because MESSAGE-GLOBIOM is the marker
scenario for this SSP. Notably, emissions from multiple sectors tend to increase
in SSP2, thus testing the harmonization mechanism with a increasing emission
pathway. Additionally, the SSP-45 scenario is chosen, where the ``45''
designation refers to a scenario with end-of-century emissions resulting in
approximately a 4.5 $\frac{\text{W}}{\text{m}^2}$ warming effect by the end of
the century, due to both warming from Greenhouse Gases (GHGs) and fluoridated
gases as well as cooling from aerosols. In this scenario, mitigation
technologies and policies are enacted causing a general reduction in pollutants
and GHGs, including (eventual) negative CO2 emissions in some regions and
sectors due to carbon capture and sequestration and afforestation. This scenario
explores the harmonization mechanism's response to generally decreasing emission
trajectories.

MESSAGE-GLOBIOM includes a representation of 11 distinct regions which can be
mapped directly to the 5-region definition used in the RCPs. Historical data is
taken from CEDS and LUC \TODO{get correct name}, which comprise 10 separate
pollutant and GHG species and 13 sectors. Therefore, a total of 1430 distinct
trajectories were harmonized.

The effect of harmonization overrides is analyzed by harmonizing each scenario
first without overrides and then with overrides. Of the 1430 trajectories,
approximately 2\% required the use of harmonization overrides after an initial
investigation; thus, 98\% of all trajectories were satisfactorily determined
using the default methods as chosen by the harmonization algorithm. Figure
\ref{fig:nox} provides one such example of a emissions species and sector in
which all regions were satisfactorily harmonized with the default methods. As
can be seen, results with and without overrides are identical.

\begin{figure}
  \begin{center}
    \includegraphics[width=\textwidth]{results_NOx_Energy_Sector.pdf}
    \caption[]{
      \label{fig:nox}
      NOx Energy Sector harmonized (solid lines) and unharmonized (dashed lines)
      trajectories for SSP2 and SSP2-45 are presented. SSP-45 is denoted with
      markers. Panel \textbf{a} shows 5-Region trajectories for scenarios
      \textit{without} any overrides. Panel \textbf{b} shows 5-Region
      trajectories for scenarios \textit{with} overrides. Panels \textbf{c} and
      \textbf{d} show global trajectories without and with overrides,
      respectively.  }
  \end{center}
\end{figure}

The harmonization of emissions pathways is performed in order to accurately
represent an updated historical emissions trajectory while also maintaining
consistency with the original, unharmonized pathway. When the default methods as
provided by the harmonization procedure distort or otherwise sufficiently
misrepresent the underlying unharmonized results, an override method is required
to be provided for the trajectory of the region, sector, and species in
question. Based on the 2\% of total trajectories required to be overridden, two
classifications were observed: regional trajectories whose \textit{magnitude}
was overly distorted and regional trajectories whose \textit{shape} was overly
distorted.

Figure \ref{fig:co} presents a situation in which the magnitude of a trajectory
is distorted. A large discrepancy (~300\% relative difference) is observed in
the harmonization year for carbon monoxide (CO) emissions in the industrial
sector specifically for the South Asia (SAS) MESSAGE-GLOBIOM region, which
comprises most of the Asian subcontinent. The default method chosen
(\code{constant_ratio} maintains model trends for the region, overall model
results are distorted. By applying a \code{constant_offset} override, the
regional trend and shape is maintained. With the new harmonization method for
the SAS region, the global trajectory for industrial CO also corresponds more
closely with the unharmonized trajectory.

\begin{figure}
  \begin{center}
    \includegraphics[width=\textwidth]{results_CO_Industrial_Sector.pdf}
    \caption[]{
      \label{fig:co}
      CO Industrial Sector harmonized and unharmonized emissions are presented
      for SSP2 and SSP2-45 scenarios. Scenarios as denoted identically to Figure
      \ref{fig:nox}. Panels \textbf{a} and \textbf{b} show harmonized and
      overridden-harmonized (respectively) regional trajectories for the 3
      MESSAGE-GLOBIOM regions that comprise the R5ASIA region: Centrally Planned
      Asia (CPA), Other Pacific Asia (PAS), and South Asia (SAS). Notably, the
      SAS regional trajectory displays a distorted trajectory due to the
      harmonization-year difference between history and model results in both
      scenarios. The distortion is large enough to affect global results, as
      shown in Panels \textbf{c} and \textbf{d}.  
}
  \end{center}
\end{figure}

In certain circumstances, the application of the default harmonization methods
can affect not only the magnitude but also the shape of regional
trajectories. Figure \ref{fig:nh3} shows emissions trajectories for ammonia
(NH3) resulting from the agriculture sector in Asia. Again, the SAS region shows
a large discrepancy in the harmonization year (> 150\% in this case). The
resulting trajectory harmonized with a \code{constant_ratio} method by default
provides a large increase after 2080 in the SSP2 reference scenario. Notably,
the SSP2-45 scenario is not affected to the same degree. While this distortion
affects the magnitude of the SAS trajectory, it largely affects the post-2080
shape of the global trajectory (see Figure \ref{fig:nh3}, panel \textbf{c}). By
using a \code{constant_offset} method as an override, this distortion is
addressed and more accurately reflects unharmonized results both in the SAS
region and global results for agricultural ammonia emissions.

\begin{figure}
  \begin{center}
    \includegraphics[width=\textwidth]{results_NH3_Agriculture.pdf}
    \caption[]{
      \label{fig:nh3}
      NH3 Agriculture harmonized and unharmonized emissions are presented for
      SSP2 and SSP2-45 scenarios. Scenarios and panel layouts are identical to
      Figure \ref{fig:co}. In this case, the SAS trajectory again shows not only
      a magnitude distortion, but also a shape distortion at the tail of the
      trajectory. Override methods have been applied to correct the distortion.
    }
  \end{center}
\end{figure}


\subsection{Discussion \& Future Work}


