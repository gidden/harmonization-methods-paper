\section{Discussion \& Future Work}\label{sec:future}

This work presented a novel methodology and Python implementation of automated
emissions harmonization for IAMs. An in-depth explanation of the processes and
methods for determining the use of harmonization methods was provided in Section
\ref{sec:meths}. The \code{aneris} code base was able to satisfactorily
harmonized over 98\% of the 2860 individual trajectories that were analyzed in
% * <maarten.vandenberg@pbl.nl> 2017-07-03T23:07:25.083Z:
% 
% > 98\%
% Do we know to what extent does this compare to other model results?
% 
% ^.
Section \ref{sec:results}. Of the remaining trajectories, harmonization method
overrides were applied, and example situations in overrides were needed
were discussed.

The automated approach drastically reduces the need for expert opinion in
determining harmonization methods for each individual combination of model
region, sector, and emissions species while still providing a justifiable
explanation for each automated choice of harmonization method based on both the
historical and future emissions trajectories. Furthermore, the automated
approach continues to scale well as models become more detailed in both the
regional and sectoral dimensions. Finally, expert opinion is still allowed to
trump the automated method as determined by the algorithm via method overrides;
however, these cases are clearly documented via the meta data provided as an
output of \code{aneris} and thus can be individually explained. This provides
not only transparency and reproducibility, but also scientific integrity in the
choice of harmonization methods.

The use of an open-source, automated harmonization process also provides
benefits to the wider climate science and IAM communities. For example, other
modeling teams are easily capable of executing identical harmonization methods
in order to participate in ongoing and further iterations of intercomparison
exercises and analysis. Future scenario analyses can also utilize this common
harmonization approach such that they are consistent with prior efforts.

There are a variety of avenues for future improvement of both the \code{aneris}
code base and underlying methodology. As with any software project, additional
users will provide use cases for more robust handling of input/output concerns
and corner cases. Further configuration parameters may also be added in the
future in order to provide overrides for all gas species in a given sector or
region. Perhaps the most fruitful investigation will involve further refinement
of the default decision tree introduced in Section \ref{sec:meths}. A key aspect
missing from the decision tree is input from models regarding whether missing
sources or activity levels are the likely cause of a harmonization year
discrepancy (suggesting the
% * <maarten.vandenberg@pbl.nl> 2017-07-03T23:15:16.273Z:
% 
% > missing
% > sources
% it might also be a difference in activity levels, which might be something different than missing sources.
% 
% ^.
use of an offset method) or instead a significant difference in emissions
% * <maarten.vandenberg@pbl.nl> 2017-07-03T23:16:40.654Z:
% 
% > emissions
% > factors (suggesting the use of a ratio method)
% knowledge of these factors might even lead to combinations of harmonization methods for a single sector/gas/region combination as they are aggregate categories.
% 
% ^.
factors (suggesting the use of a ratio method) \cite{rogelj_discrepancies_2011}.

However, this work provides a new direction and framework which the IAM and
climate communities can build upon in order to reduce the necessity of ad-hoc
expert opinion and increase transparency and reproducibility of harmonization
exercises. Furthermore, it provides an open-source, tested, and documented code
base which can be used and improved upon by these communities. Both of these are
clear steps in a positive direction for future climate and integrated assessment
modeling exercises.
