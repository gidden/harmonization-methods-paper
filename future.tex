\section{Discussion \& Future Work}\label{sec:future}

This work presented a novel methodology and Python implementation of automated
emissions harmonization for IAMs. An in-depth explanation of the processes and
methods for determining the use of harmonization methods was provided in Section
\ref{sec:meths}. The \code{aneris} codebase was able to satisfactorily
harmonized over 98\% of the 2860 individual trajectories that were analyzed in
Section \ref{sec:results}. Of the remaining trajectories, harmonization method
overrides were applied, and the situation in which the need for overrides arose
was discussed.

The automated approach critically removes the need for expert opinion in
determining harmonization methods for each individual combination of model
region, sector, and emissions species while still providing a justified reason
for each automated choice of harmonization method based on both the historical
and future emissions trajectories. Furthermore, the automated approach continues
to scale well as models become more detailed in both the regional and sectoral
dimensions. Finally, expert opinion is still allowed to trump the automated
method as determined by the algorithm via method overrides; however, these cases
are clearly documented via the metadata provided as an output of \code{aneris}
and thus can be individually explained. This provides not only transparency but
also scientific integrity in the choice of harmonization methods.

There are a variety of avenues for future improvement of both the \code{aneris}
codebase and underlying methdology. As with any software project, additional
users will provide use cases for more robust handling of input/output concerns
and corner cases. Further configuration parameters may also be added in the
future in order to provide overrides for all gas species in a given sector or
region. Perhaps the most fruitful investigation will involve further refinement
of the defaul decision tree introduce in Section \ref{sec:meths}. A key aspect
missing from the decision tree is input from models regarding whether missing
sources are the likely cause of a harmonization year discrepancy (suggesting the
use of an offset method) or instead a discrepancy in emissions factors
(suggesting the use of a ratio method)\TODO{cite Joeri}. 

However, this work provides a new direction and framework in which the IAM and
climate communities can follow in order to reduce the necessicity of expert
opinion and increase transparency and reproducibility of harmonization
exercises. Furthermore, it provides an open-source, tested, and documented code
base which can be used and improved upon by these communities. Both of these are
clear steps in a positive direction for future climate and integrated assesment
modeling exercises.
