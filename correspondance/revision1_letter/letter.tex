%%%%%%%%%%%%%%%%%%%%%%%%%%%%%%%%%%%%%%%%%
% Plain Cover Letter
% LaTeX Template
%
% This template has been downloaded from:
% http://www.latextemplates.com
%
% Original author:
% Rensselaer Polytechnic Institute (http://www.rpi.edu/dept/arc/training/latex/resumes/)
%
%%%%%%%%%%%%%%%%%%%%%%%%%%%%%%%%%%%%%%%%%

%----------------------------------------------------------------------------------------
%       PACKAGES AND OTHER DOCUMENT CONFIGURATIONS
%----------------------------------------------------------------------------------------

\documentclass[11pt]{letter} % Default font size of the document, change to 10pt to fit more text
\usepackage{graphicx}
%\usepackage{newcent} % Default font is the New Century Schoolbook PostScript font
%\usepackage{helvet} % Uncomment this (while commenting the above line) to use the Helvetica font

% Margins
\usepackage[left=1.25in,right=1.25in,top=1.5in,bottom=1.25in]{geometry}
%\let\raggedleft\raggedright % Pushes the date (at the top) to the left, comment this line to have the date on the right

\usepackage{eso-pic,graphicx}
 \begin{document}

%----------------------------------------------------------------------------------------
%       ADDRESSEE SECTION
%----------------------------------------------------------------------------------------

\begin{letter}{
    Professor S. Reis\\
    Editor, Environmental Modelling \& Software\\
    \hfill\break
    Professor D.P. Ames\\
    Editor-in-Chief, Environmental Modelling \& Software\\
  }

%----------------------------------------------------------------------------------------
%       YOUR NAME & ADDRESS SECTION
%----------------------------------------------------------------------------------------

\address{Matthew Gidden\\
Laimgrubengasse 17/7\\
Vienna, Austria 1060}


%----------------------------------------------------------------------------------------
%       LETTER CONTENT SECTION
%----------------------------------------------------------------------------------------

\opening{Professors Reis and Ames,}

Please find enclosed our revised manuscript and response to reviewers regarding
the submitted manuscript entitled: ``A Methodology and Implementation of
Automated Emissions Harmonization for Use in Integrated Assessment Models''
(Ref: \texttt{ENVSOFT\_2017\_668}).

We very much appreciate the time and thoroughness taken by the three reviewers
of our manuscript, and believe that the resulting revised manuscript is
non-trivially improved over the submitted version. While we have provided direct
responses to each point made by the reviewers in the attached response document,
we summarize their primary suggestions below:

\begin{itemize}
\item We should better explain the intuition behind the harmonization method choices
\item We should greatly expand on the implementation details of the software
\item We should clarify and add a listing of all used acronyms
\item We should provide more context and sources of further reading for various terms and models discussed in the text.
\item We should consider framing the results section as a case study.
\end{itemize}

We have fully addressed and taken on board these core suggestions in the revised
manuscript. In order to better prime the discussion in the \textit{Methodology \&
  Implementation} section, we have written an introductory subsection entitled
\textit{The Conceptual Basis for Choosing Harmonization Methods} which outlines
the intuition for various methods to be used in the harmonization process. We
have greatly expanded the \textit{Python Implementation and Workflow} subsection
to include specific details of the software implementation, code snippets and
CLI listings, and a new figure (Figure 5 in the text) which provides a graphical
description of the implementation.

We have additionally made use of a global acronyms table which identifies and
lists all acronyms used in the text and have placed this table at the end of the
manuscript for readers' reference. In all instances raised by the reviewers for
further clarification on terms and models, we have added footnotes with an
additional description and links or references for further reading on the
topic. Finally, we have reframed slightly the last section, renaming it to
\textit{Case Study: Harmonizing Results from a Global IAM}. We agree with
Reviewer 3 that even in its initial form it is in fact a study of a specific use
case with \texttt{aneris} that details both the positive aspects (that the vast
majority of trajectories were harmonized satisfactorily) as well as what we did
to solve specific cases where default methods led to non-preferable harmonized
results.

After revision, we believe that the manuscript has been improved over the
initially submitted draft. The authors would like to thank the reviewers for
their constructive comments and suggestions as well as for their time taken to
review our article. We would also again like to thank you again for your
consideration of this work.

\closing{
Warm Regards,

Matthew Gidden, Ph.D.
}

%----------------------------------------------------------------------------------------

\end{letter}

\end{document}


